\documentclass{article}
\usepackage{amsmath,amsthm,amsfonts,amssymb,bm}

\begin{document}
Consider the Inverse Model, 
\begin{equation*}
X_{y} = \mu + \Gamma\nu_{y}+ \varepsilon,
\end{equation*}
where $X_{y}\in\mathbb{R}^{p},\nu_{y}\in\mathbb{R}^{d},d<p,$ the basis $\Gamma\in\mathbb{R}^{p\times d}$ with $\Gamma^{T}\Gamma=I_{d}$, and $\varepsilon\sim N_{p}(0,\sigma^2I_{p}).$

The following proposition states under the assumption of inverse model, $\Gamma$ is actually a sufficient reduction. See Cook(2007) for more general case.

\textbf{Proposition:} Under the inverse model, the distribution of $Y|X$ is the same as the distribution of $Y|\Gamma^{T}X$.

\textbf{Proof:} Firstly, $X|Y=y\sim N_{p}(\mu+\Gamma\nu_{y},\sigma^2I_{p})$. By Bayesian formula, we have
\begin{align*}
f_{Y|X}(y|x) &\propto f_{X|Y}(x|y)f_{Y}(y)\\
 		   &\propto exp(-\frac{1}{2\sigma^2}\|x-\mu-\Gamma\nu_{y}\|^{2})f_{Y}(y)\\
		   &\propto exp(-\frac{1}{2\sigma^2}(\nu_{y}^{T}\nu_{y}-2\nu_{y}^{T}\Gamma^{T}(x-\mu))f_{Y}(y)
\end{align*}
The last line is given by the orthogonality of $\Gamma$. Similarly,
since $\Gamma^{T}X|Y=y\sim N_{d}(\Gamma^{T}\mu + \nu_{y},\sigma^2I_{d})$, we have
\begin{align*}
f_{Y|\Gamma^{T}X}(y|\Gamma^{T}x) &\propto f_{\Gamma^{T}X|Y}(\Gamma^{T}x|y)f_{Y}(y)\\
			&\propto exp(-\frac{1}{2\sigma^2}\|\Gamma^{T}x-\Gamma^{T}\mu-\nu_{y}\|^{2})f_{Y}(y)\\
			&\propto exp(-\frac{1}{2\sigma^2}(\nu_{y}^{T}\nu_{y}-2\nu_{y}^{T}\Gamma^{T}(x-\mu))f_{Y}(y)
\end{align*}
Therefore, the kernel of $Y|X$ and $Y|\Gamma^{T}X$ are the same, which implies the result.




\end{document}