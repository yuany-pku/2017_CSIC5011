%!TEX encoding = UTF-8 Unicode
%\documentclass[slidestop,compress,epsfig,color]{beamer}
\documentclass[slidestop,epsfig,color]{beamer}
\mode<presentation>{}
\usetheme{Copenhagen}
\usepackage{pstricks}
\usepackage{graphicx}
\usepackage{amsmath,amssymb,amsthm}
\usepackage{beamerthemesplit}
\useoutertheme[subsection=false]{smoothbars} 
%\useoutertheme{sidebar} 
%\useoutertheme{default}
\useoutertheme{miniframes} %miniframes
\useinnertheme{rectangles}
\usepackage{CJKutf8}

%\beamersetuncovermixins{\opaqueness<1>{25}}{\opaqueness<2->{15}}

\providecommand{\C}{\mathbb{C}}
\providecommand{\R}{\mathbb{R}}
\providecommand{\X}{\mathcal{X}}
\providecommand{\Y}{\mathcal{Y}}
\providecommand{\Z}{\mathbb{Z}}
\providecommand{\Q}{\mathbb{Q}}
\providecommand{\E}{\mathbb{E}}
\providecommand{\H}{\mathcal{H}}
\providecommand{\D}{\mathcal{D}}
\providecommand{\U}{\mathcal{U}}
\providecommand{\De}{\Delta}
\providecommand{\de}{\delta}
\providecommand{\hg}{\hat{g}}

\theoremstyle{example}
\newtheorem{thm}{}
%\newtheorem{fact}{}

\providecommand{\subitem}{\\ \textcolor{yellow}{$\bullet\ $}}
\DeclareMathOperator{\sign}{sign}
%\DeclareMathOperator{\span}{span}
\DeclareMathOperator{\supp}{supp}
%\DeclareMathOperator{\ker}{ker}
\DeclareMathOperator{\im}{im}
\DeclareMathOperator{\proj}{proj}
\DeclareMathOperator{\grad}{grad}
\DeclareMathOperator{\curl}{curl}
\DeclareMathOperator{\dive}{div}
\DeclareMathOperator{\ASL}{\mathfrak{sl}}



\title[HKUST, Fall 2017]{An Introduction to Topological Data Analysis}
\author{Yuan Yao}
\institute{Department of Mathematics \\ HKUST}
\date{November, 2017}
\addtobeamertemplate{title page}{}{}

\begin{document}

\frame{\titlepage}

\section[Outline]{}
\frame{\tableofcontents}

\section[Why]{Why Topological Methods?}

\subsection{Methods for Visualizing a Data Geometry}
\frame{
\frametitle{Methods for Summarizing or Visualizing a Geometry}
\begin{figure}[!h]
\centering
\includegraphics[width=0.4\textwidth]{figures/geom_project.pdf}  \\
\caption[ ]{Linear projection (PCA, MDS, variable selection, etc)}
\end{figure}
}

\frame{
\frametitle{Methods for Summarizing or Visualizing a Geometry}
\begin{figure}[!h]
\centering
\includegraphics[width=0.6\textwidth]{figures/geom_nldr.png}  \\
\caption[ ]{Nonlinear Dimensionality Reduction (ISOMAP, LLE etc.)}
\end{figure}
}

\frame{
 \frametitle{Geometric Data Reduction}
 \begin{itemize}
 \item General method of \textcolor{red}{manifold learning} takes the following \textcolor{red}{Spectral Kernal Embedding} approach
 \subitem construct a \textcolor{blue}{neighborhood graph} of data, $G$
 \subitem construct a \textcolor{blue}{positive semi-definite kernel} on graphs, $K$
 \subitem find global embedding coordinates of data by \textcolor{blue}{eigen-decomposition} of $K=YY^T$
 \item Sometimes `distance metric' is just a similarity measure (nonmetric MDS, ordinal embedding)
 \item Sometimes coordinates are not a good way to organize/visualize the data (e.g. $d>3$) 
 \item Sometimes all that is required is a \textcolor{red}{qualitative} view
 \end{itemize}
}


\subsection{Why Topology?}

\frame{
\frametitle{Topology}
\begin{itemize}
\item \small Origins of Topology in Math
\subitem Leonhard Euler 1736, Seven Bridges of Königsberg
\subitem Johann Benedict Listing 1847, Vorstudien zur Topologie
\subitem J.B. Listing (orbituary) Nature 27:316-317, 1883.
“qualitative geometry from the ordinary geometry in which quantitative relations chiefly are treated.” 
\end{itemize}
%\begin{itemize}
%\item We would like to say that all points within \textcolor{red}{tolerance} are the same
%\item Moreover, all non-zero distances beyond \textcolor{red}{tolerance} are the same, i.e. invariant under distortion
%\end{itemize}
\begin{figure}[!h]
\centering
\includegraphics[width=0.8\textwidth]{figures/seven_bridges.png} 
\end{figure}
}

\frame{
  \frametitle{RNA hairpin folding pathways}
\begin{figure}[!h] 
\begin{center}
\begin{tabular}{c}
\centerline{\includegraphics[width=0.55\textwidth]{figures/mapper_RFCE10_8l.pdf}} 
\end{tabular}
\caption{Jointly with Xuhui Huang, Jian Sun, Greg Bowman, Gunnar Carlsson, Leo Guibas, and Vijay Pande, \emph{JACS'08}, \emph{JCP'09}}
\end{center}
\end{figure}
}

\frame{
  \frametitle{Progression of Breast Cancer with gene expression profiles}
\begin{figure}[!h] 
\begin{center}
\begin{tabular}{c}
\centerline{\includegraphics[width=0.8\textwidth]{figures/breastcancer.pdf}} 
\end{tabular}
\caption{Monica Nicolau, A. Levine, and Gunnar Carlsson, \emph{PNAS'10}}
\end{center}
\end{figure}
}

\frame{
	\frametitle{Key elements}
	\begin{itemize}
	\item Coordinate free representation
	\item Invariance under deformations
	\item Compressed qualitative representation
	\end{itemize}
	}
	
\frame{
\frametitle{Topology}
\begin{itemize}
\item To see points in neighborhood the \emph{same} requires distortion of distances, i.e. stretching and shrinking
\item We do not permit \emph{tearing}, i.e. distorting distances in a discontinuous way
\end{itemize}
}


\frame{
\frametitle{Continous Topology}
\begin{figure}[!h]
\centering
\includegraphics[width=0.9\textwidth]{figures/knots.png}
\caption{Homeomorphic} 
\end{figure}
}

\frame{
\frametitle{Continuous Topology}
\begin{figure}[!h]
\centering
\includegraphics[width=0.9\textwidth]{figures/sphere.png}
\caption{Homeomorphic} 
\end{figure}
}

\frame{
	\frametitle{Discrete case?}
\centering{\emph{How does topology make sense, in \textcolor{red}{discrete} and \textcolor{red}{noisy} setting?}}
}

%\subsection{Methods for Imposing or Visualizing a Data Geometry}

%\frame{
%\frametitle{Methods for Imposing a Geometry}
%\begin{figure}[!h]
%\centering
%\includegraphics[width=0.7\textwidth]{figures/geom_metric.pdf}  \\
%\caption[ ]{Define a metric}
%\end{figure}
%}

%\frame{
%\frametitle{Methods for Imposing a Geometry}
%\begin{figure}[!h]
%\centering
%\includegraphics[width=0.6\textwidth]{figures/geom_network.pdf}  \\
%\caption[ ]{Define a graph or network structure}
%\end{figure}
%}
%
%
%\frame{
%\frametitle{Methods for Imposing a Geometry}
%\begin{figure}[!h]
%\centering
%\includegraphics[width=0.6\textwidth]{figures/geom_cluster.pdf}  \\
%\caption[ ]{Clustering the data}
%\end{figure}
%}
%
%
%\frame{
%\frametitle{Methods for Summarizing or Visualizing a Geometry}
%\begin{figure}[!h]
%\centering
%\includegraphics[width=0.3\textwidth]{figures/geom_tree.png}  \\
%\caption[ ]{Project to a Tree (Evolution and Phylogenetics)}
%\end{figure}
%}

%\subsection{Properties of Data Geometry}

\frame{
\frametitle{Properties of Data Geometry}
\begin{fact}
We Don't Trust Large Distances!
\end{fact}
 \begin{itemize}
 \item In life or social sciences, \textcolor{red}{distance (metric)} are constructed using a notion of \textcolor{red}{similarity (proximity)}, but have no theoretical backing (e.g. distance between faces, gene expression profiles, Jukes-Cantor distance between sequences)
 \item Small distances still represent similarity (proximity), but long distance comparisons hardly make sense
 \end{itemize} 
}

\frame{
\frametitle{Properties of Data Geometry}
\begin{fact}
We Only Trust Small Distances a Bit!
\end{fact}
\begin{figure}[!h]
\centering
\includegraphics[width=0.4\textwidth]{figures/small_distance.png} 
\end{figure}
 \begin{itemize}
 \item Both pairs are regarded as similar, but the strength of the similarity as encoded by the distance may not be so significant
 \item Similar objects lie in neighborhood of each other, which suffices to define \textcolor{red}{topology}
 \end{itemize} 
}

\frame{
\frametitle{Properties of Data Geometry}
\begin{fact}
Even Local Connections are Noisy, depending on observer's scale!
\end{fact}
\begin{figure}[!h]
\centering
\includegraphics[width=0.4\textwidth]{figures/circle_dots.jpg} 
\end{figure}
 \begin{itemize}
 \item Is it a circle, dots, or circle of circles?
 \item To see the circle, we ignore variations in small distance (tolerance for proximity)
 \end{itemize} 
}

\frame{
\frametitle{So we need topology for robustness against metric distortions}
\begin{itemize}
\item Distance measurements are noisy
\item Physical device like human eyes may ignore differences in proximity (or as an average effect)
 \item \textcolor{red}{Topology} is the crudest way to capture invariants under distortions of distances
 \item At the presence of \textcolor{red}{noise}, one need \textcolor{red}{topology varied with scales}
\end{itemize}
} 

\frame{
\frametitle{What kind of topology?}
\begin{itemize}
\item Topology studies (global) mappings between spaces 
\item \textcolor{blue}{Point-set} topology: continuous mappings on open sets
\item \textcolor{blue}{Differential} topology: differentiable mappings on smooth manifolds
\subitem Morse theory tells us topology of continuous space can be learned by discrete information on critical points
\item \textcolor{blue}{Algebraic} topology: homomorphisms on algebraic structures, the most concise encoder for topology
\item \textcolor{blue}{Combinatorial} topology: mappings on \textcolor{red}{simplicial (cell) complexes}
\subitem simplicial complex may be constructed from data
\subitem Algebraic, differential structures can be defined here
\end{itemize}
}

\frame{
\frametitle{Topological Data Analysis}
\begin{itemize}
\item What kind of topological information often useful
\subitem 0-homology: clustering or connected components
\subitem 1-homology: coverage of sensor networks; paths in robotic planning
\subitem 1-homology as obstructions: inconsistency in statistical ranking; harmonic flow games
\subitem high-order homology: high-order connectivity?
\item How to compute homology in a stable way?
\subitem \emph{simplicial complexes} for data representation
\subitem \emph{filtration} on simplicial complexes
\subitem \emph{persistent homology}
\end{itemize}
}

\frame{
\frametitle{Betti Numbers: the number of $i$-dim holes}
\begin{figure}[!h]
\includegraphics[width=0.8\textwidth]{figures/betti_circle.png}  
%\caption{The birth and death of connected components.}
\end{figure}
}

\frame{
\frametitle{Betti Numbers: the number of $i$-dim holes}
\begin{figure}[!h]
%\centering
\includegraphics[width=0.4\textwidth]{./figures/betti_sphere0.png}  
\caption{Sphere: $\beta_0=1$, $\beta_1=0$, $\beta_2=1$, and $\beta_k=0$ for $k\geq 3$}
\end{figure}
}

\frame{
\frametitle{Betti Numbers: the number of $i$-dim holes}
\begin{figure}[!h]
%\centering
\includegraphics[width=0.8\textwidth]{./figures/betti_torus.png}  
%\caption{The birth and death of connected components.}
\end{figure}
}

\frame{
	\frametitle{Betti Numbers and Homology Groups}
	\begin{itemize}
	\item Betti numbers are computed as dimensions of Boolean vector spaces (E. Noether, $\Z_2$-homology group) \pause
	\item $\beta_i(X)=dim H_i(X, \Z_2)$, $\Z_2$-homology or more general Homology group associated with any fields or integral domain (e.g. $\Z$, $\Q$, and $\R$) \pause
	\item $H_i(X)$ is \emph{functorial}, i.e. continuous mapping $f: X\to Y$ induces linear transformation $H_i(f): H_i(X) \to H_i(Y)$, structure preserving \pause
	\item computation is simple linear algebra over fields or integers \pause
	\item data representation by \emph{simplicial complexes}  
	\end{itemize}
	}
	
\section[Simplicial Complex]{Simplicial Complex for Data Representation}

\subsection{Simplicial Complex}
\frame{
\frametitle{Simplicial Complexes for Data Representation}
\begin{definition}[Simplicial Complex]
An abstract simplicial complex is a collection $\Sigma$ of subsets of $V$ which is closed under inclusion (or deletion), i.e. $\tau\in\Sigma$ and $\sigma\subseteq \tau$, then $\sigma\in \Sigma$. 
\end{definition}
\begin{itemize}
\item Chess-board Complex
\item Term-document cooccurance complex
\item Nerve complex
\item Point cloud data in metric spaces:
\subitem \v{C}ech, Rips, Witness complex
\subitem Mayer-Vietoris Blowup
\item Clique complex in pairwise comparison graphs
\item Strategic complex in game theory
\end{itemize}
}

\frame{
\frametitle{Chess-board Complex}
\begin{definition}[Chess-board Complex]
Let $V$ be the positions on a Chess board. $\Sigma$ collects position subsets of $V$ where one can place queens (rooks) without capturing each other. 
\end{definition}
\begin{itemize}
\item Closedness under deletion: if $\sigma\in \Sigma$ is a set of ``safe'' positions, then any subset $\tau\subseteq\sigma$ is also a set of ``safe'' positions  
\end{itemize}
\begin{figure}[!h]
\centering
\includegraphics[width=0.2\textwidth]{figures/chessboard-queen-moves.png} \ \ \
\includegraphics[width=0.2\textwidth]{figures/queen8.jpg} 
\end{figure}
}

\frame{
\frametitle{Term-Document Co-occurrence Complex}
\begin{figure}[!h]
%\centering
\includegraphics[width=0.3\textwidth]{./figures/cooccurrence_mat.pdf}  
\includegraphics[width=0.4\textwidth]{./figures/cooccurrence.pdf} 
\end{figure}
\begin{itemize}
\item Left is a term-document co-occurrence matrix
\item Right is a simplicial complex representation of terms
\item Connectivity analysis captures more information than Latent Semantic Index (Li \& Kwong 2009)
\end{itemize}
}

\subsection{Nerve, Reeb Graph, and Mapper}
\frame{
\frametitle{Nerve complex}
\begin{definition}[Nerve Complex]
Define a cover of $X$, $X=\cup_\alpha U_\alpha$. $V=\{U_\alpha\}$ and define $\Sigma=\{U_I: \cap_{\alpha\in I} U_I \neq \emptyset \}$. 
\end{definition}
\begin{itemize}
\item Closedness under deletion
\item Can be applied to any topological space $X$
%\item In a metric space $(X,d)$, if $U_\alpha = B_\epsilon(t_\alpha):=\{x\in X: d(x-t_\alpha)\leq \epsilon\}$, we have \textcolor{red}{\v{C}ech complex} $C_\epsilon$. 
\item \textcolor{red}{Nerve Theorem}: if every $U_I$ is contractible, then $X$ has the same homotopy type as $\Sigma$.  
\end{itemize}
}

\frame{
	\frametitle{Nerve complex example}
\begin{figure}[!h]
%\centering
\includegraphics[width=0.4\textwidth]{./figures/circle_cover.pdf}  
\caption{Covering of circle}
\end{figure}
}

\frame{
	\frametitle{Nerve complex example}
\begin{figure}[!h]
%\centering
\includegraphics[width=0.4\textwidth]{./figures/circle_cover.pdf}  
\includegraphics[width=0.4\textwidth]{./figures/circle_nodes.pdf}  
\caption{Create nodes}
\end{figure}
}

\frame{
	\frametitle{Nerve complex example}
\begin{figure}[!h]
%\centering
\includegraphics[width=0.4\textwidth]{./figures/circle_cover.pdf}  
\includegraphics[width=0.4\textwidth]{./figures/circle_nerve.pdf}  
\caption{Create edges, that gives a Nerve complex (graph)}
\end{figure}
}

\frame{
	\frametitle{Nerve of Seven Bridges of Königsberg}
%\begin{itemize}
%\item We would like to say that all points within \textcolor{red}{tolerance} are the same
%\item Moreover, all non-zero distances beyond \textcolor{red}{tolerance} are the same, i.e. invariant under distortion
%\end{itemize}
\begin{figure}[!h]
\centering
\includegraphics[width=\textwidth]{figures/seven_bridges_nerve.pdf} 
\caption{Nerve graph of Seven Bridges of Könisberg}
\end{figure}
}


\frame{
	\frametitle{Point cloud data}
\begin{itemize}
\item Now given point cloud data $\X=\{x_1,\ldots, x_n\}$, and a covering $V=\{U_\alpha\}$, where each $U_\alpha$ is a cluster of data
\item Build a simplicial complex (Nerve) in the same way, but components replaced by clusters  
\end{itemize}
}

\frame{
	\frametitle{Mapping}
\begin{itemize}
\item How to choose coverings?
\item Create a reference map (or filter) $h: \X\to \mathcal{Z}$, where $\mathcal{Z}$ is a topological space often with interesting metrics (e.g. $\R$, $\R^2$, $S^1$ etc.), and a covering $\mathcal{U}$ of $\mathcal{Z}$, then construct the covering of $\X$ using inverse map $\{h^{-1} U_\alpha\}$. 
\end{itemize}
}

\frame{
  \frametitle{Example: Morse Theory and Reeb graph}
 \begin{itemize}
 \item a nice (Morse) function: $h: \X \to \R$, on a smooth manifold $\X$
 \item topology of $\X$ reconstructed from level sets $h^{-1}(t)$ 
 \item topological of $h^{-1}(t)$ only changes at `\textcolor{red}{critical values}' 
 \item \textcolor{red}{Reeb graph}: a simplified version, contracting into points the connected components in $h^{-1}(t)$
 \end{itemize}
\begin{figure}[t]
\centering
\begin{tabular}{c}
	\includegraphics[width=0.4\textwidth]{figures/Reeb_graph.pdf}  
\end{tabular}
	\caption{Construction of Reeb graph; $h$ maps each point on torus to its height.
	\label{torus}} 
\end{figure}
}

\frame{
  \frametitle{Mapper: from Continuous to Discrete...}
\begin{figure}[t]
\centering
\begin{tabular}{c}
	\includegraphics[width=0.4\textwidth]{figures/Mapper_graph.pdf}  
\end{tabular}
	\caption{An illustration of Mapper.} 
\end{figure}
Note: 
\begin{itemize}
\item degree-one nodes contain local minima/maxima;
\item degree-three nodes contain saddle points (critical points);
\item degree-two nodes consist of regular points 
\end{itemize}
}

\frame{
  \frametitle{Mapper algorithm}
[{\it\tiny Singh-Memoli-Carlsson. Eurograph-PBG, 2007}] Given a data set $\X$,
  
  \begin{itemize}
  \item choose a \textcolor{red}{filter} map $h:\X\to \mathcal{Z}$, where $\mathcal{Z}$ is a topological space such as $\R$, $S^1$, $\R^d$, etc.
  \item choose a cover $\mathcal{Z} \subseteq \cup_\alpha U_\alpha$ 
  \item \textcolor{red}{cluster/partite} level sets $h^{-1}(U_\alpha)$ into $V_{\alpha, \beta}$ 
  \item \textcolor{red}{graph} representation: a node for each $V_{\alpha,\beta}$, an edge between $(V_{\alpha_1,\beta_1},V_{\alpha_2,\beta_2})$ iff $U_{\alpha_1} \cap U_{\alpha_2} \neq \emptyset$ and $V_{\alpha_1,\beta_1}\cap V_{\alpha_2,\beta_2}\neq \emptyset$.
  \item extendable to \textcolor{red}{simplicial complex representation}.  
  \end{itemize}
 
 \medskip
 
 Note: it extends \textcolor{blue}{Reeb Graph} from $\R$ to general topological space $\mathcal{Z}$; may lead to a particular implementation of \textcolor{blue}{Nerve theorem} through filter map $h$. 
}

\frame{
  \frametitle{In applications.}
  Reeb graph has found various applications in computational geometry, statistics under different names.
  \begin{itemize}
  \item computer science: contour trees, Reeb graphs
  \item statistics: density cluster trees (Hartigan)
  \end{itemize}
  \begin{figure}
	\centering
	\includegraphics[width=0.6\textwidth]{figures/density-tree.pdf}
%	\caption{Non-Gaussian Clusters}
\end{figure}
} 

\frame{
  \frametitle{Reference Mapping}
  Typical one dimensional filters/mappings:
  \begin{itemize}
  \item Density estimators
  \item Measures of data (ec-)centrality: e.g. $\sum_{x^\prime \in \X} d(x,x^\prime)^p$
  \item Geometric embeddings: PCA/MDS, Manifold learning, Diffusion Maps etc.
  \item Response variable in statistics: progression stage of disease etc.
  \end{itemize}
} 

\frame{
  \frametitle{Example: RNA Tetraloop}

\begin{columns} 
\begin{column}{0.3\textwidth} 
\begin{figure}[h]
\centering
	\includegraphics[width=0.9\textwidth]{figures/gcaa.pdf} \\
	\includegraphics[width=0.4\columnwidth]{figures/native_contactmap_sm.png}
	\caption{RNA GCAA-Tetraloop} 
\end{figure}
\end{column} 
\begin{column}{0.7\textwidth} 
Biological relevance: 

\begin{itemize}
\item serve as nucleation site for RNA folding
\item form sequence specific tertiary interactions
\item protein recognition sites
\item certain Tetraloops can pause RNA transcription
\end{itemize}
Note: simple, but, \textcolor{red}{biological debates over intermediate states} on folding pathways 
\end{column} 
\end{columns} 
}

\frame{
  \frametitle{Debates: Two-state vs. Multi-state Models}

\begin{columns}
\begin{column}{0.3\textwidth}
\begin{figure}[!h]
%\centering
\begin{tabular}{c}
	\includegraphics[width=0.7\textwidth]{figures/state2.pdf}  \\
	(a) 2-state model \\
	\includegraphics[width=0.7\textwidth]{figures/state3.pdf} \\
	(b) multi-state model
\end{tabular}
%	\caption{(a) 2-state model; (b) multi-state model} 
\end{figure}
\end{column}
\begin{column}{0.7\textwidth}
\begin{itemize}
\item 2-state: transition state with any one stem base pair, from \textcolor{red}{thermodynamic} experiments  [{\tiny \it Ansari A, et al. PNAS, 2001, 98: 7771-7776}]
\item multi-state: there is a stable intermediate state,
which contains collapsed structures, from \textcolor{red}{kinetic} measurements [{\tiny \it Ma H, et al. PNAS, 2007, 104:712-6}]
\item experiments: \textcolor{red}{no} structural information
\item computer simulations at full-atom resolution:
\subitem \textcolor{red}{exisitence} of intermediate states
\subitem if yes, what's the \textcolor{red}{structure}?
\end{itemize}
\end{column}
\end{columns}
}

\frame{
  \frametitle{Mapper with density filters in biomolecular folding}
Reference: \textcolor{blue}{Bowman-Huang-Yao et al. J. Am. Chem. Soc. 2008; Yao, Sun, Huang, et al. J. Chem. Phys. 2009.}
\begin{itemize}
\item \textcolor{red}{densest} regions (energy basins) may correspond to \textcolor{red}{metastates} (e.g. folded, extended) 
\item \textcolor{red}{intermediate/transition states} on pathways connecting them are \textcolor{red}{relatively sparse} 
\end{itemize}

Therefore with Mapper 
\begin{itemize}
\item \textcolor{red}{clustering on density level sets} helps separate and identify metastates and intermediate/transition states
\item \textcolor{red}{graph} representation reflects kinetic connectivity between states 
\end{itemize} 
}

\frame{
  \frametitle{A vanilla version}
  \begin{figure}
	\centering
	\includegraphics[width=0.9\textwidth]{figures/mapper.pdf}
	\caption{Mapper Flow Chart}
  \end{figure}
 \begin{enumerate}
 \item Kernel density estimation $h(x)= \sum_i K(x,x_i)$ with Hamming distance for contact maps
 \item Rank the data by $h$ and divide the data into $n$ overlapped sets 
 \item Single-linkage clustering on each level sets
 \item Graphical representation
 \end{enumerate}
}

\frame{
  \frametitle{Mapper output for Unfolding Pathways}
  \begin{figure}[!h] 
\begin{center}
\begin{tabular}{c}
\centerline{\includegraphics[width=0.7\textwidth]{figures/mapper_UFCE10_8l.pdf}} 
\end{tabular}
\caption{Unfolding pathway}
\end{center}
\end{figure}
}


\frame{
  \frametitle{Mapper output for Refolding Pathways}
\begin{figure}[!h] 
\begin{center}
\begin{tabular}{c}
\centerline{\includegraphics[width=0.6\textwidth]{figures/mapper_RFCE10_8l.pdf}} 
\end{tabular}
\caption{Refolding pathway }
\end{center}
\end{figure}
}

\frame{
  \frametitle{Progression of Breast Cancer: $l_2$-eccentrality}
\begin{figure}[!h] 
\begin{center}
\begin{tabular}{c}
\centerline{\includegraphics[width=0.8\textwidth]{figures/breastcancer.pdf}} 
\end{tabular}
\caption{Monica Nicolau, A. Levine, and Gunnar Carlsson, \emph{PNAS'10}}
\end{center}
\end{figure}
}

\frame{
  \frametitle{Cell Cycles}
\begin{figure}[!h] 
\begin{center}
\begin{tabular}{c}
\centerline{\includegraphics[width=0.4\textwidth]{figures/cellcycle.pdf}} 
\end{tabular}
\caption{Cell Cycle Microarray Data, courtesy of M. Nicolau, Nagarajan, G. Singh, Carlsson}
\end{center}
\end{figure}
}

\frame{
  \frametitle{Relationships between diabetic, pre-diabetic, and healthy populations}
\begin{figure}[!h] 
\begin{center}
\begin{tabular}{c}
\centerline{\includegraphics[width=0.5\textwidth]{figures/diabetes.pdf}} 
\end{tabular}
\caption{Miller-Reaven Diabetes Dataset, courtesy of Gunnar Carlsson}
\end{center}
\end{figure}
}

\frame{
  \frametitle{Leukemia with gene expression profiles}
\begin{figure}[!h] 
\begin{center}
\begin{tabular}{c}
\centerline{\includegraphics[width=0.5\textwidth]{figures/leukemia.pdf}} 
\end{tabular}
\caption{Topological structure of Leukemia: courtesy of Gunnar Carlsson}
\end{center}
\end{figure}
}

\subsection{\v{C}ech, Vietoris-Rips, and Witness Complexes}
\frame{
\frametitle{\v{C}ech complex}
\begin{definition}[\v{C}ech Complex $C_\epsilon$]
In a metric space $(X,d)$, define a cover of $X$, $X=\cup_\alpha U_\alpha$ where $U_\alpha = B_\epsilon(t_\alpha):=\{x\in X: d(x-t_\alpha)\leq \epsilon\}$. $V=\{U_\alpha\}$ and define $\Sigma=\{U_I: \cap_{\alpha\in I} U_I \neq \emptyset \}$. 
\end{definition}
\begin{itemize}
\item Closedness under deletion
\item Can be applied to any metric space $X$
\item \textcolor{red}{Nerve Theorem}: if every $U_I$ is contractible, then $X$ has the same homotopy type as $\Sigma$.  
\end{itemize}
}

\frame{
\frametitle{Example: \v{C}ech Complex}
\begin{figure}[!h]
\centering
\includegraphics[width=0.9\textwidth]{figures/nerve.png} 
\caption{\v{C}ech complex of a circle, $C_\epsilon$, covered by a set of balls.}
\end{figure}
}

\frame{
\frametitle{Vietoris-Rips complex}
\begin{itemize}
\item \v{C}ech complex is hard to compute, even in Euclidean space
\item One can easily compute an upper bound for \v{C}ech complex
\subitem Construct a \v{C}ech subcomplex of 1-dimension, i.e. a graph with edges connecting point pairs whose distance is no more than $\epsilon$. 
\subitem Find the clique complex, i.e. maximal complex whose 1-skeleton is the graph above, where every $k$-clique is regarded as a $k-1$ simplex 
\end{itemize}
\begin{definition}[Vietoris-Rips Complex]
Let $V=\{x_\alpha\in X\}$. Define $VR_\epsilon = \{U_I \subseteq V: d(x_\alpha,x_\beta)\leq \epsilon, \alpha,\beta\in I\}$.  
\end{definition}
}

\frame{
\frametitle{Example: Rips Complex}
\begin{figure}[!h]
\centering
\includegraphics[width=0.9\textwidth]{figures/rips_cech.png} 
\caption{Left: \v{C}ech complex gives a circle; Right: Rips complex gives a sphere $S^2$.}
\end{figure}
}

\frame{
\frametitle{Generalized Vietoris-Rips for Symmetric Relations}
\begin{definition}[Symmetric Relation Complex]
Let $V$ be a set and a symmetric relation $R=\{(u,v)\} \subseteq V^2$ such that $(u,v)\in R\Rightarrow (v,u)\in R$. $\Sigma$ collects subsets of $V$ which are in pairwise relations. 
\end{definition}

\begin{itemize}
\item Closedness under deletion: if $\sigma\in \Sigma$ is a set of related items, then any subset $\tau\subseteq\sigma$ is a set of related items
\item Generalized Vietoris-Rips complex beyond metric spaces
\item E.g. Zeeman's tolerance space  
\item C.H. Dowker defines simplicial complex for unsymmetric relations
\end{itemize}
}

\frame{
\frametitle{Sandwich Theorems}
\begin{itemize}
\item Rips is easier to compute than Cech
\subitem even so, Rips is exponential to dimension generally
\item However Vietoris-Rips CAN NOT preserve the homotopy type as Cech
\item But there is still a hope to find a \textcolor{red}{lower bound} on homology --
\end{itemize}
\begin{theorem}[``Sandwich'']
$$  VR_{\epsilon} \subseteq {C}_{\epsilon} \subseteq VR_{2\epsilon} $$ 
\end{theorem}
\begin{itemize}
\item If a homology group ``persists'' through $R_{\epsilon} \to R_{2\epsilon}$, then it must exists in $C_{\epsilon}$; but not the vice versa. 
\end{itemize}
}



\frame{
\frametitle{A further simplification: Witness complex}
\begin{definition}[Strong Witness Complex]
Let $V=\{t_\alpha\in X\}$. Define $W^{s}_\epsilon = \{U_I \subseteq V: \exists x\in X, \forall \alpha\in I, d(x,t_\alpha)\leq \textcolor{red}{d(x,V)} + \epsilon\}$.  
\end{definition}
\begin{definition}[Week Witness Complex]
Let $V=\{t_\alpha\in X\}$. Define $W^{w}_\epsilon = \{U_I \subseteq V: \exists x\in X, \forall \alpha\in I, d(x,t_\alpha)\leq \textcolor{red}{d(x,V_{-I} )} + \epsilon\}$.  
\end{definition}
\begin{itemize}
\item $V$ can be a set of landmarks, much smaller than $X$
\item Monotonicity: $W^*_\epsilon \subseteq W^*_{\epsilon'}$ if $\epsilon\leq \epsilon'$
\item But not easy to control homotopy types between $W^*$ and $X$
\end{itemize}
}

\frame{
\frametitle{Strategic Simplicial Complex for Flow Games}
\begin{figure}[!h]
%\centering
\includegraphics[width=0.3\textwidth]{./figures/battleSex_mat.pdf}  
\includegraphics[width=0.3\textwidth]{./figures/battleSex.pdf} 
\end{figure}
\begin{itemize}
\item Strategic simplicial complex is the clique complex of pairwise comparison graph above, inspired by ranking
\item Every game can be decomposed as the direct sum of potential games and zero-sum games (harmonic games) (Candogan, Menache, Ozdaglar and Parrilo 2010)
\end{itemize}
}

\section{Persistent Homology}

\subsection{Betti Number at Different Scales}
\frame{
\frametitle{Example I: Persistent Homology of \v{C}ech Complexes}
\begin{figure}[!h]
%\centering
\includegraphics[width=0.4\textwidth]{./figures/persistent1.png} 
\includegraphics[width=0.4\textwidth]{./figures/persistent3.png}   
\caption{Scale $\epsilon_1$: $\beta_0=1$, $\beta_1=3$}
\end{figure}
}

\frame{
\frametitle{Example I: Persistent Homology of \v{C}ech Complexes}
\begin{figure}[!h]
%\centering
\includegraphics[width=0.4\textwidth]{./figures/persistent2.png}  
\includegraphics[width=0.4\textwidth]{./figures/persistent4.png}  
\caption{Scale $\epsilon_1$: $\beta_0=1$, $\beta_1=2$ }
\end{figure}
}

\frame{
\frametitle{Example II: Persistence 0-Homology induced by Height Function}
\begin{figure}[!h]
%\centering
\includegraphics[width=0.9\textwidth]{./figures/persistence_diagram.jpg}  
\caption{The birth and death of connected components.}
\end{figure}
}

\frame{
\frametitle{Example III: Persistent Homology as Online Algorithm to Track Topology Changements}
\begin{figure}[!h]
%\centering
\includegraphics[width=0.9\textwidth]{./figures/persistence.jpg}  
\caption{The birth and death of simplices.}
\end{figure}
}

\frame{
\frametitle{Persistent Betti Numbers: Barcodes}
\begin{figure}[!h]
%\centering
\includegraphics[width=1\textwidth]{./figures/Barcode_dim0.png}  
\end{figure}
\begin{figure}[!h]
%\centering
\includegraphics[width=1\textwidth]{./figures/Barcode_dim1.png}  
\end{figure}
\begin{itemize}
\item Toolbox: JavaPlex (\url{https://github.com/appliedtopology/javaplex/wiki/Tutorial})
\subitem Java version of Plex, work with matlab
\subitem Rips, Witness complex, Persistence Homology
\item Other Choices: Plex 2.5 for Matlab (not maintained any more), Dionysus (Dimitry Morozov)
\end{itemize}
}


\subsection{Algebraic Theory}
\frame{
\frametitle{Persistent Homology: Algebraic Theory [Zormorodian-Carlsson]}
\begin{itemize}
\item All above gives rise to a filtration of simplicial complex
$$ \emptyset = \Sigma_0 \subseteq \Sigma_1 \subseteq \Sigma_2 \subseteq \ldots $$
\item Functoriality of inclusion: there are homomorphisms between homology groups 
$$ 0 \rightarrow H_1 \rightarrow H_2 \rightarrow \dots $$
\item A persistent homology is the image of $H_i$ in $H_j$ with $j>i$. 
\end{itemize}
}


\frame{
\frametitle{Persistent 0-Homology of Rips Complex}
\begin{itemize}
\item Equivalent to \textcolor{red}{single-linkage} clustering
\item Barcode is the single linkage dendrogram (tree) without labels 
\item Kleinberg's Impossibility Theorem for clustering: no clustering algorithm satisfies scale invariance, richness, and consistency
\item Memoli \& Carlsson 2009: single-linkage is the unique \textcolor{red}{persistent clustering} with scale invariance
\item \textcolor{blue}{Open:} but, is persistence the necessity for clustering? 
\item Notes: try matlab command \textcolor{blue}{\texttt{linkage}} for single-linkage clustering.
\end{itemize}
}

%\subsection{Some Applications}

\subsection{Application: Sensor Network Coverage}
\frame{
\frametitle{Application I: Sensor Network Coverage by Persistent Homology}
\begin{itemize}
\item \textcolor{blue}{V. de Silva and R. Ghrist  (2005)} Coverage in sensor networks via persistent homology. 
\item Ideally sensor communication can be modeled by Rips complex
\subitem two sensors has distance within a short range, then two sensors receive strong signals;
\subitem two sensors has distance within a middle range, then two sensors receive weak signals;
\subitem otherwise no signals 
\end{itemize}
}

\frame{
\frametitle{Sandwich Theorem}
\begin{theorem}[de Silva-Ghrist 2005] Let $X$ be a set of points in $R^d$
and $C_\epsilon(X )$ the \v{C}ech complex of the cover of
$X$ by balls of radius $\epsilon/2$. Then there is chain of inclusions
$$ R_{\epsilon^\prime}(X ) \subset C_\epsilon(X ) \subset R_\epsilon (X ) \ \  \textit{whenever} \ \ \ 
\frac{\epsilon}{\epsilon^\prime} \geq \sqrt{\frac{2d}{d+1}}. $$ 
Moreover, this ratio is the smallest for which the inclusions hold in general.
\end{theorem}
\textcolor{red}{Note}: this gives a sufficient condition to detect holes in sensor network coverage
\begin{itemize}
\item \v{C}ech complex is hard to compute while Rips is easy; 
\item If a hole persists from $R_{\epsilon^\prime}$ to $R_\epsilon$, then it must exists in $C_\epsilon$.  
\end{itemize}
}

\frame{
\frametitle{Persistent 1-Homology in Rips Complexes}
\begin{figure}[!h]
%\centering
\includegraphics[width=0.5\textwidth]{./figures/rips1.png}  
\includegraphics[width=0.5\textwidth]{./figures/rips2.png}  
\caption{Left: $R_{\epsilon^\prime}$; Right: $R_\epsilon$. The middle hole persists from $R_{\epsilon^\prime}$ to $R_\epsilon$.}
\end{figure}
}

\subsection{Application: Genetic Recombination}
\frame{
	\frametitle{Genome-wide Maps of Human Recombination}
\begin{figure}[!h]
%\centering
\includegraphics[width=0.7\textwidth]{./figures/raul2.pdf}  
%\caption{Prof. WEI, Guowei at MSU, \emph{SIAM News 2017}}
\end{figure}
}

\frame{
	\frametitle{Persistent $\beta_1$ associated with recombination rates}
\begin{figure}[!h]
%\centering
\includegraphics[width=0.7\textwidth]{./figures/raul1.pdf}  
%\caption{Prof. WEI, Guowei at MSU, \emph{SIAM News 2017}}
\end{figure}
}


\subsection{Application: Biomolecular Structure}
\frame{
	\frametitle{Persistent Homology Analysis of Biomolecular Data}
\begin{figure}[!h]
%\centering
\includegraphics[width=0.6\textwidth]{./figures/guowei.pdf}  
\caption{Prof. WEI, Guowei at MSU, \emph{SIAM News 2017}}
\end{figure}
}

\frame{
	\frametitle{Persistent Homology Analysis of Biomolecular Data}
\begin{itemize}
\item Persistent Homology as Barcodes provides multiscale analysis of protein 3D structure 
\item Combined with machine learning (deep learning, random forests, boosting), it provides best free energy ranking for Set 1 (Stage 2) in D3R Grand Challenge 2, a worldwide competition in computer-aided drug design (\url{http://bit.ly/2h4Vm6q})
\end{itemize}
}


\subsection{Application: Natural Image Patches}
\frame{
\frametitle{Application: Natural Image Statistics}
\begin{itemize}
\item \textcolor{blue}{G. Carlsson, V. de Silva, T. Ishkanov, A. Zomorodian} (2008) On the local behavior of spaces of natural images, \emph{International Journal of Computer Vision}, 76(1):1-12.
\item An image taken by black and white digital camera can be
viewed as a vector, with one coordinate for each pixel
\item  Each pixel has a “gray scale” value, can be thought of as a real number (in reality, takes one of 255 values)
\item Typical camera uses tens of thousands of pixels, so images lie in a very high dimensional space, call it pixel space, $\mathcal{P}$
\end{itemize}
}

\frame{
\frametitle{Natural Image Statistics}
\begin{itemize}
\item {\textbf{D. Mumford}}: What can be said about the set of images $\mathcal{I} \subseteq  \mathcal{P}$ one obtains when one takes many images with a digital camera?
\item {\textbf{Lee, Mumford, Pedersen}}: Useful to study \textcolor{red}{local} structure of images statistically
\end{itemize}
}

\frame{
\frametitle{Natural Image Statistics}
\begin{figure}[!h]
%\centering
\includegraphics[width=0.5\textwidth]{./figures/patch3x3.png}  
\caption{$3\times 3$ patches in images}
\end{figure}
}

\frame{
\frametitle{Natural Image Statistics}
Lee-Mumford-Pedersen [LMP] study only high contrast patches.
\begin{itemize}
\item Collect: $4.5 M$ high contrast patches from a collection of images obtained by van Hateren and van der Schaaf
\item Normalize mean intensity by subtracting mean from each pixel value to obtain patches with mean intensity = 0
\item Puts data on an 8-D hyperplane, $\approx R^8$
\item Furthermore, normalize contrast by dividing by the norm, so obtain patches with norm = 1, whence data lies on a 7-D ellipsoid, 
$\approx S^7$
\end{itemize}
}

\frame{
\frametitle{Natural Image Statistics: Primary Circle}
High density subsets $\mathcal{M}(k=300,t=0.25)$:
\begin{itemize}
\item Codensity filter: $d_k(x)$ be the distance from $x$ to its $k$-th nearest neighbor
\subitem the lower $d_k(x)$, the higher density of $x$ 
\item Take $k=300$, the extract $5,000$ top $t=25\%$ densest points, which concentrate on a \textcolor{red}{primary circle}
\end{itemize}
\begin{figure}[!h]
%\centering
\includegraphics[width=0.5\textwidth]{./figures/mumford_k300t25.png}  
%\caption{$3\times 3$ patches in images}
\end{figure}
}

\frame{
\frametitle{Natural Image Statistics: Three Circles}
\begin{itemize}
\item Take $k=15$, the extract $5,000$ top $25\%$ densest points, which shows persistent $\beta_1=5$, 3-circle model
\end{itemize}
\begin{figure}[!h]
%\centering
\includegraphics[width=0.7\textwidth]{./figures/mumford_k15t25.png}  
%\caption{$3\times 3$ patches in images}
\end{figure}
}

\frame{
\frametitle{Natural Image Statistics: Three Circles}
Generators for 3 circles
\begin{figure}[!h]
%\centering
%\includegraphics[width=0.7\textwidth]{./figures/mumford_k300t25_1circle.png}  
\includegraphics[width=0.7\textwidth]{./figures/mumford_3circle.png}  
%\caption{$3\times 3$ patches in images}
\end{figure}
}


\frame{
\frametitle{Natural Image Statistics: Klein Bottle}
\begin{figure}[!h]
%\centering
\includegraphics[width=0.7\textwidth]{./figures/mumford_klein.png}  
%\caption{$3\times 3$ patches in images}
\end{figure}
}

\frame{
\frametitle{Natural Image Statistics: Klein Bottle Model}
\begin{figure}[!h]
%\centering
\includegraphics[width=0.6\textwidth]{./figures/mumford_klein_model.png}  
%\caption{$3\times 3$ patches in images}
\end{figure}
}

%\subsection{Molecular Dynamics}
%\frame{
%\frametitle{Application III: Persistent Homology and Discrete Morse Theory}
%\begin{itemize}
%\item Persistent homology gives a pairing (birth-death) between a simplex and its co-dimensional one faces  
%\item It leads to a particular implementation of Robin Forman's combinatorial gradient field
%\item Thus Persistent homology is equivalent to \textcolor{red}{discrete Morse Theory by Robin Forman}
%\end{itemize}
%}
%
%\frame{
%  \frametitle{Morse Theory and Reeb graph}
% \begin{itemize}
% \item a nice (Morse) function: $h: \X \to \R$, on a smooth manifold $\X$
% \item topology of $\X$ reconstructed from level sets $h^{-1}(t)$ 
% \item topological of $h^{-1}(t)$ only changes at `\textcolor{red}{critical values}' 
% \item \textcolor{red}{Reeb graph}: a simplified version, contracting into points the connected components in $h^{-1}(t)$
% \end{itemize}
%\begin{figure}[t]
%\centering
%\begin{tabular}{c}
%	\includegraphics[width=0.4\textwidth]{figures/Reeb_graph.pdf}  
%\end{tabular}
%	\caption{Construction of Reeb graph; $h$ maps each point on torus to its height.
%	\label{torus}} 
%\end{figure}
%}
%
%\frame{
%  \frametitle{In applications.}
%  Reeb graph has found various applications in computational geometry, statistics under different names.
%  \begin{itemize}
%  \item computer science: contour trees, reeb graphs
%  \item statistics: density cluster trees (Hartigan)
%  \end{itemize}
%  \begin{figure}
%	\centering
%	\includegraphics[width=0.6\textwidth]{figures/density-tree.pdf}
%%	\caption{Non-Gaussian Clusters}
%\end{figure}
%} 
%
%\frame{
%  \frametitle{Mapper: an extension for topological data analysis}
%[{\it\tiny Singh-Memoli-Carlsson. Eurograph-PBG, 2007}] Given a data set $\X$,
%  
%  \begin{itemize}
%  \item choose a \textcolor{red}{filter} map $h:\X\to T$, where $T$ is a topological space such as $\R$, $S^1$, $\R^d$, etc.
%  \item choose a cover $T \subseteq \cup_\alpha U_\alpha$ 
%  \item \textcolor{red}{cluster/partite} level sets $h^{-1}(U_\alpha)$ into $V_{\alpha, \beta}$ 
%  \item \textcolor{red}{graph} representation: a node for each $V_{\alpha,\beta}$, an edge between $(V_{\alpha_1,\beta_1},V_{\alpha_2,\beta_2})$ iff $U_{\alpha_1} \cap U_{\alpha_2} \neq \emptyset$ and $V_{\alpha_1,\beta_1}\cap V_{\alpha_2,\beta_2}\neq \emptyset$.
%  \item extendable to \textcolor{red}{simplicial complex representation}.  
%  \end{itemize}
% 
% \medskip
% 
% Note: it extends \textcolor{blue}{Morse theory} from $\R$ to general topological space $T$; may lead to a particular implementation of \textcolor{blue}{Nerve theorem} through filter map $h$. 
%}
%
%\frame{
%  \frametitle{An example with real valued filter}
%\begin{figure}[t]
%\centering
%\begin{tabular}{c}
%	\includegraphics[width=0.4\textwidth]{figures/Mapper_graph.pdf}  
%\end{tabular}
%	\caption{An illustration of Mapper.} 
%\end{figure}
%Note: 
%\begin{itemize}
%\item degree-one nodes contain local minima/maxima;
%\item degree-three nodes contain saddle points (critical points);
%\item degree-two nodes consist of regular points 
%\end{itemize}
%}


%\subsection{Progression Analysis of Disease}
%\frame{
%  \frametitle{Application IV: Progression Analysis for Breast Cancer}
%\begin{itemize}
%\item \textcolor{blue}{Nicolau, Levine, Carlsson}, \emph{PNAS}, 2010
%\item Deviation functions from normal tissues are used as filters (Morse-type functions)
%\item Mapper (Reeb Graph) with such filters leads to Progression Analysis of Disease
%\end{itemize}
%}
%
%
%\frame{
%  \frametitle{PAD analysis of the NKI data}
%\begin{figure}[!h] 
%\begin{center}
%\begin{tabular}{c}
%\centerline{\includegraphics[width=0.9\textwidth]{figures/nicolau.png}} 
%\end{tabular}
%%\caption{Nicolau, Levine, Carlsson, PNAS, 2010 }
%\end{center}
%\end{figure}
%}
%


%\frame{
%  \frametitle{Transition Counts: 2ps lag time}
% \begin{figure}
%	\centering
%	\includegraphics[width=0.8\textwidth]{figures/transit9_yangjian.pdf}
%\end{figure}
%\begin{itemize}
%\item The two intermediate states, are on-pathways; the inner base-pair formation
%is easier in proceeding than backing ($.15/.07$), while the end base-pair formed more reluctant ($.12/.09)$  
%\item Note that this is not a Markov State Model.
%\end{itemize} 
%}



%\frame{
%\frametitle{From Topology to Geometry: Hodge Theory}
%\begin{itemize}
%\item When one studies rational homology, Laplacians can be defined  
%\item Kernels of Laplacians, i.e. harmonics, gives a particular basis to represent rational homology
%\item Such a basis often gives more information about the metrics, or geometry of simplicial complex
%\item In general, filtrations can be reflected as metric information in Laplacians
%\item \textcolor{red}{Open}: filtrations converted to discrete Morse functions, then Witten-Morse deformations on Laplacians
%\end{itemize}
%}




%
%\section[Morse Method in MD]{Molecular Dynamics and Morse Theory}
%
%\frame{
%  \frametitle{Molecular Dynamics and Morse Theory}
%  Collaborators:
%  \begin{itemize}
%  \item Biology/Chemistry: \textcolor{red}{Xuhui Huang, Greg Bowman, Vijay Pande}
%  \item Computer Science: \textcolor{red}{Jian Sun, Leo Guibas}
%  \item Mathematics: \textcolor{red}{Michael Lesnick, Gurjeet Singh, Gunnar Carlsson}
%  \end{itemize}
%  
%  Thanks to
%  \begin{itemize}
%  \item Michael Levitt (Stanford)
%  \item John Chodera (UCB)
%  \item Wing-Hung Wong (Stanford)
%  \item Nancy Zhang (Stanford)
%  \end{itemize}
%}
%
%\subsection{Challenges in RNA Haripin Folding}
%
%
%\frame{
%  \frametitle{SREMD Simulations}
%  
%\begin{columns} 
%\begin{column}{0.22\textwidth} 
%\begin{figure}[h]
%\centering
%\begin{tabular}{c}
%	\includegraphics[width=1.0\textwidth]{figures/foldingathome0.pdf} \\
%%	\caption{Over 250,000 CPUs now participate in Folding$@$home }
%	\includegraphics[width=0.9\textwidth]{figures/box.pdf}  \\
%	Simulation Box.
%\end{tabular}
%\end{figure}
%\end{column} 
%\begin{column}{0.78\textwidth} 
%[{\tiny \it Bowman, Huang, Y., Sun, ... Vijay. JACS, 2008}] 
%\smallskip
%\begin{itemize}
%\item 2800 SREMD (Serial Replica Exchange Molecular Dynamics) simulations with RNA hairpin (5'-GGGCGCAAGCCU-3')
%\item 389 RNA atoms, $\sim$4000 water and 11 $Na^+$ 
%\item SREMD random walks in temperature space (56 ladders from 285K to 646K) with molecular dynamic trajectories
%\item 210,000 ns simulations with $\sim$105,000,000 configurations 
%\item Unfortunately, sampling still \textcolor{red}{not converged}!
%\end{itemize}
%\end{column} 
%\end{columns} 
%}
%
%\frame{
% \frametitle{Challenges for Data Analysis}
%\begin{itemize}
%\item Massive volume and high dimensionality: $100M$ samples in $12K$ Cartesian coordinates
%$\Rightarrow$ \textcolor{blue}{contact maps} as $55$-bit string
%\item \textcolor{red}{Looking for a needle in a haystack}: 
%\subitem intermediates/transition states of interests are of \textcolor{blue}{low-density} 
%\subitem folded/unfolded states are dominant
%\item \textcolor{red}{Samples are not in equilibrium distribution} 
%\begin{figure}[t] 
%\begin{tabular}{cc}
%\includegraphics[width=0.2\columnwidth]{figures/gcaa.pdf} &
%\includegraphics[width=0.2\columnwidth]{figures/native_contactmap.pdf}\\
%(a)&(b)
%\end{tabular}
%\caption{(a) NMR structure of the GCAA tetraloop.  (b) Contact map.}
%\end{figure}
%\end{itemize}
%}
%
%\frame{
% \frametitle{Needle through magnifying glasses: Conditional distributions}
% Conditioning on the region where intermediate states may host:
%\begin{itemize}
%\item folding/unfolding events
%\begin{columns}
%\begin{column}{0.6\textwidth}
%\begin{figure}[h]
%\centering
%\includegraphics[width=0.8\columnwidth]{figures/events.pdf}
%\end{figure}
%\end{column}
%\begin{column}{0.4\textwidth}
%\begin{itemize}
%\item 760 unfolding events;
%\item 550 folding events;
%\end{itemize}
%\end{column}
%\end{columns} 
%\item biased toward the target states (folded/extended)
%\end{itemize}
%Note: applicable to non-equilibrium distributed data.
%}
%
%\frame{
%  \frametitle{Our strategy}
%\textcolor{red}{Problem:} How to separate sparse intermediates from dense folded/unfolded structures? 
%\begin{solution} 
%\center{\textcolor{red}{stratify data into density level sets, and} }
%\center{\textcolor{red}{cluster on each level set} }
%\end{solution}
%
%\medskip
%
%\textcolor{red}{But}, can we organize those clusters in a systematic way? 
%\begin{itemize}
%\item Yes, \textcolor{blue}{Morse theory} in mathematics provides an inspiration...
%\end{itemize}
%}
%
%\subsection{Morse Theory, Reeb Graph, and Mapper}
%
%
%\frame{
%  \frametitle{Transition Counts: 2ps lag time}
% \begin{figure}
%	\centering
%	\includegraphics[width=0.8\textwidth]{figures/transit9_yangjian.pdf}
%\end{figure}
%\begin{itemize}
%\item The two intermediate states, are on-pathways; the inner base-pair formation
%is easier in proceeding than backing ($.15/.07$), while the end base-pair formed more reluctant ($.12/.09)$  
%\item Note that this is not a Markov State Model.
%\end{itemize} 
%}

%\frame{
%  \frametitle{Biological Suggestions from Mapper Results}
%  [{\tiny \it Bowman, et al. JACS 2008}]
%  \begin{itemize}
% \item Folding and unfolding follows \textcolor{red}{different pathways}
% \item For folding pathways, there are \textcolor{red}{multiple intermediate states}
% \subitem a dominant one with inner/closing stem base-pair formed
% \subitem a less dominant one with outer/end stem base-pair formed
% \item This in the first time provides \textcolor{red}{structural evidence} from simulations in support of multistate hypothesis on folding pathways
%  \end{itemize}
%  Reproducible research: data and codes to reproduce those figures are available at
%  
%  \textcolor{blue}{\tt https://simtk.org/home/rna-mapper}
%}
%
%\frame{
% \frametitle{Future direction: Transition Networks}
% \begin{figure}
%	\centering
%	\includegraphics[width=0.7\textwidth]{figures/MRTN.pdf}
% \end{figure}
%}
%
%\section[Hodge Rank]{Rank Aggregation and Hodge Theory}
%
%\frame{
%  \frametitle{Rank Aggregation and Hodge Theory}
%    Collaborators:
%  \begin{itemize}
%  \item Mathematics: \textcolor{red}{Lek-Heng Lim} (UCB)
%  \item ICME and Computer Science: \textcolor{red}{Xiaoye Jiang} (student)
%  \item Management Science and Engineering: \textcolor{red}{Yinyu Ye}
%  \end{itemize}
%  
%  Thanks to
%  \begin{columns}
%  \begin{column}{0.47\textwidth}
%  \begin{itemize}
%  \item Gunnar Carlsson (Stanford) 
%  \item Vin de Silva (Pomona)
%  \item Persi Diaconis (Stanford) 
%  \item Leo Guibas (Stanford)
%  \item Fei Han (Stanford)
%  \end{itemize}
%  \end{column}
%  \begin{column}{0.53\textwidth}
%  \begin{itemize}
%  \item Ming Ma (Beijing Institute of Technology) 
%  \item Art Owen (Stanford)
%  \item Steve Smale (TTI-C \& UCB) 
%  \item Don Saari (UCI)
%  \item Bin Yu (UCB)
%  \end{itemize}
%  \end{column}
%  \end{columns}
%}
%
%\frame{
%\frametitle{Hodge decomposition}
%
%\begin{itemize}
%\item \textbf{Vector calculus:} Helmholtz's decomposition, ie.\ vector fields on
%\textit{nice domains} may be resolved into irrotational (curl-free) and
%solenoidal (divergence-free) component vector fields
%\[
%\mathbf{F}=-\nabla\varphi+\nabla\times\mathbf{A}%
%\]
%$\varphi$ scalar potential, $\mathbf{A}$ vector potential.
%
%\item \textbf{Linear algebra:} additive orthogonal decomposition of a
%skew-symmetric matrix into three skew-symmetric matrices
%\[
%W = W_{1} + W_{2} + W_{3}
%\]
%$W_{1} = \mathbf{v}\mathbf{e}^{T} - \mathbf{e} \mathbf{v}^{T}$, $W_{2}$
%clique-consistent, $W_{3}$ inconsistent.
%
%\item \textbf{Graph theory:} orthogonal decomposition of network flows into
%acyclic and cyclic components.
%\end{itemize}
%}
%
%\frame{ 
%\frametitle{Ranking on networks (graphs)}
%\begin{itemize}
%\item Multicriteria rank/decision systems
%
%\begin{itemize}
%\item Amazon or Netflix's recommendation system (user-product)
%
%\item Interest ranking in social networks (person-interest)
%
%\item S\&P index (time-price)
%
%\item Voting (voter-candidate)
%\end{itemize}
%
%\item Peer review systems
%
%\begin{itemize}
%\item publication citation systems (paper-paper)
%
%\item Google's webpage ranking (web-web)
%
%\item eBay's reputation system (customer-customer)
%\end{itemize}
%\end{itemize}
%}
%
%\frame{
%\frametitle{Characteristics}
%Aforementioned ranking data often
%
%\begin{itemize}
%\item incomplete: typically about 1\% 
%
%\item imbalanced: heterogeneously distributed votes 
%
%\item cardinal: given in terms of scores or stochastic choices
%\end{itemize}
%
%Implicitly or explicitly, ranking data may be viewed to live on a simple graph
%$G=(V,E)$, where
%
%\begin{itemize}
%\item $V$: set of alternatives (products, interests, etc) to be ranked
%
%\item $E$: pairs of alternatives to be compared
%\end{itemize}
%}
%
%\frame{
%  \frametitle{Example I: Netflix Customer-Product Rating}
%
%  \begin{example}[Netflix Customer-Product Rating]
%  \begin{itemize}
%  \item $480189$-by-$17770$ customer-product rating matrix $X$
%  \item $X$ contains $98.82\%$ missing values
%  \end{itemize}
%  \end{example}
%
% However,
%
%\begin{itemize}
%\item pairwise comparison graph $G=(V,E)$ is very \textcolor{red}{dense}!
%
%\item only $0.22\%$ edges are missed, \textcolor{red}{almost a complete graph}
%
%\item rank aggregation may be carried out without estimating missing values
%
%\item \textcolor{red}{imbalanced}: number of raters on $e\in E$ varies
%\end{itemize}
%
%\textbf{Caveat:} we are not trying to solve the Netflix prize problem
%}
%
%\frame{
%\frametitle{Netflix example continued}
%
%The \textbf{first order} statistics, mean score for each product, is often
%inadequate because of the following:
%
%\begin{itemize}
%\item most customers would rate just a \textcolor{red}{very small portion} of the products
%
%\item different products might have different raters, whence mean scores
%involve noise due to \textcolor{red}{arbitrary individual rating scales}
%\end{itemize}
%}
%
%\frame{
%\frametitle{Temporal Drifts of Netflix Mean Scores}
%\begin{columns}
%\begin{column}{0.7\textwidth}
%\begin{figure}[h]
%\centering
%\centerline{\includegraphics[width=0.8\textwidth]{figures/6movies_drifts.pdf}} 
%\end{figure}
%\end{column}
%\begin{column}{0.3\textwidth}
%\begin{itemize}
%\item How about \textcolor{red}{high order} statistics?
%\item Pairwise comparison removes such drifts. 
%\end{itemize}
%\end{column}
%\end{columns} 
%}
%
%\frame{
%  \frametitle{From $1^{st}$ Order to $2^{nd}$ Order: Pairwise Rankings}
%  \begin{itemize}
%  \item \textcolor{red}{Linear Model}: mean score difference between product $i$ and $j$ over all customers who have rated both of them,
%\[  w_{ij} = \frac{\sum_k (X_{kj} - X_{ki})}{\#\{k: X_{ki}, X_{kj}\text{ exist}\}}, \]
%is \textcolor{red}{translation invariant}.
% \item \textcolor{red}{Log-linear Model}:
%\[ w_{ij} =\frac{\sum_{k}(\log X_{kj}-\log X_{ki})}{\#\{k:X_{ki},X_{kj}\text{
%exist}\}}, \]
%is \textcolor{red}{scale invariant}.
%
%  \end{itemize}
%}
%
%\frame{
%  \frametitle{More invariant}
%
%  \begin{itemize}
%   \item \textcolor{red}{Linear Probability Model}: the probability that product $j$ is preferred to $i$ in excess of a purely random choice,
%\[  w_{ij} = \Pr \{k:X_{kj} > X_{ki}\} - \frac{1}{2}.\]
%This is \textcolor{red}{invariant up to a monotone transformation}.
% 
%  \item \textcolor{red}{Bradley-Terry Model}: logarithmic odd ratio (logit)
%\[
%w_{ij}=\log\frac{\Pr\{k:X_{kj}>X_{ki}\}}{\Pr\{k:X_{kj}<X_{ki}\}}.
%\]
%\textcolor{red}{Invariant up to monotone transformation}.
%\end{itemize}
%}
%
%\frame{
%\frametitle{Pairwise ranking graph for IMDb top 20 movies}%
%%EndExpansion
%\begin{figure}[ptb]
%\includegraphics[width=0.65\textwidth]{figures/imdb20_1.pdf}
%%\resizebox{9cm}{!}{\input{figures/Hodge1.pstex_t}}
%\caption{Pairwise rankings are network flows}%
%\end{figure}%
%}
%
%\frame{
%
%  \frametitle{Example II: Purely Exchange Economics}
%  
%  \begin{example}[Pairwise ranking in exchange market]
%  \begin{itemize} 
%  \item $n$ goods $V=\{1,\ldots,n\}$ in an exchange market, with an
%exchange rate matrix $A$, such that 
%\[ 1 \text{ unit }i = a_{ij} \text{ unit } j, \qquad a_{ij}>0 .\]
%which is a \textcolor{red}{reciprocal matrix}, i.e.\ $a_{ij}=1/a_{ji}$
%  \item Ideally, a product triple $(i,j,k)$ is called \textcolor{red}{triangular arbitrage-free}, if $a_{ij} a_{jk} = a_{ik}$
%  \item \textcolor{red}{Money} (universal equivalent): does there exist a universal equivalent with pricing function $p: V\to \R^+$, such that
%  \[ a_{ij} = p_j / p_i ?\]
%  \end{itemize}
%  \end{example}
%}
%
%\frame{
%  \frametitle{From Pairwise to Global}
%  \begin{itemize}
%    \item Under the logarithmic map, $w_{ij} = \log a_{ij}$, we have an equivalent theory:
%    \subitem the \textcolor{red}{triangular arbitrage-free} is equivalent to
%\[ w_{ij} + w_{jk} + w_{ki} = 0 \]
%  \subitem \textcolor{red}{universal equivalent} is a global ranking $q: V\to \R$ ($q_i=\log p_i$) such that
%  \[ w_{ij} = q_j - q_i  =:(\textcolor{red}{\delta_0} q)(i,j) \]
%  \item Here 
%  \subitem \textcolor{red}{Global ranking} $\Leftrightarrow$ universal equivalent (price)
%  \subitem \textcolor{red}{Pairwise ranking} $\Leftrightarrow$ exchange rates
%  \end{itemize}
%}
%
%\frame{
%  \frametitle{Observations}
%  In both examples,
%  \begin{itemize}
%  \item contain \textcolor{red}{cardinal} information 
%  \item involve \textcolor{red}{pairwise} comparisons
%  \end{itemize} 
%  \medskip
%  
%  \textcolor{blue}{How important are they?}
%}
%
%\frame{
%
%  \frametitle{Ordinal Rank Aggregation}
%
%  \begin{itemize}
%  \item \textcolor{red}{Problem}: given a set of partial/total order $\{\succeq_i: i=1,\ldots,n\}$ on a common set $V$,
%find
%\[ (\succeq_1,\ldots, \succeq_n) \mapsto \succeq^\ast, \]
%as a partial order on $V$, satisfying certain \emph{optimal} condition.
%  \item Examples: 
%  \subitem voting 
%  \subitem Social Choice Theory
%  \item \textcolor{red}{Notes}:
%  \subitem Impossibility Theorems (Arrow and Sen)
%  \subitem NP-Hardness in solving Kemeny optimization
%  \subitem Harmonic analysis on permutation group $S_n$ is impractical for large $n$ 
%  \end{itemize}
%}
%
%\frame{
%  \frametitle{Cardinal Rank Aggregation in Our Approach}
%
%\begin{problem}
%Does there exist a global ranking function, $v: V\to\mathbb{R}$, such that
%\[
%w_{ij} = v_{j} - v_{i} =: \delta_{0} (v) (i,j) ?
%\]
%
%\end{problem}
%
%Equivalently, does there exists a scalar field $v:V\rightarrow\mathbb{R}$
%whose gradient field gives the flow $w$? ie.\ is $w$ \textcolor{red}{integrable}?
%%  \begin{itemize}
%%  \item \textcolor{red}{Problem}: given a set of functions $f_i:V\to \R$ $(i=1\ldots,n)$, find 
%%  \[ (f_1,\ldots,f_n) \mapsto f^\ast \]
%%  as a function on $V$, satisfying certain \emph{optimal} condition.
%%  \item \textcolor{red}{Notes}
%%  \subitem relaxations leave rooms for `possibility' 
%%  \subitem ordinal rankings induced from cardinal rankings, but with information loss
%%  \end{itemize}
%}
%
%\frame{
%\frametitle{Answer: not always!}%
%%EndExpansion
%
%
%As in multivariate calculus where non-integrable vector fields exist, similarly we have
%
%\begin{figure}[ptb]
%\includegraphics[width=3cm]{figures/cyclic0.pdf}
%%\resizebox{3cm}{!}{\input{figures/harmonic.pstex_t}}
%\includegraphics[width=3cm]{figures/harmonic.pdf}
%\caption{No global ranking
%$v$ gives $w_{ij}=v_{j}-v_{i}$: (a) triangular cyclic, note $w_{AB}%
%+w_{BC}+w_{CA}\neq0$; (b) it contains a 4-node cyclic flow $A\rightarrow
%C\rightarrow D\rightarrow E\rightarrow A$, note on a $3$-clique $\{A,B,C\}$
%(also $\{A,E,F\}$), $w_{AB}+w_{BC}+w_{CA}=0$ }%
%\end{figure}%
%}
%
%\frame{
%\frametitle{Triangular transitivity}%
%%EndExpansion
%
%
%\begin{fact}
%$W=[w_{ij}]$ skew symmetric associated with graph $G = (V, E)$. If
%$w_{ij}=v_{j}-v_{i}$ for all $\{i,j\} \in E$, then $w_{ij} +w_{jk}+w_{ki}=0$
%for all $3$-cliques $\{i,j,k\}$.
%\end{fact}
%
%\textcolor{red}{Transitivity subspace}:
%\[
%\{W\text{ skew symmetric}\mid w_{ij}+w_{jk}+w_{ki}=0\text{ for all
%$3$-cliques}\}
%\]
%Example in the last slide, (a) lies outside; (b) lies in this subspace, but
%not a gradient flow.
%}
%
%\frame{
%\frametitle{Hodge theory: graph theoretic}%
%%EndExpansion
%
%
%\textcolor{red}{Orthogonal decomposition} of network flows on $G$ into
%\[
%\text{gradient flow} + \text{globally cyclic} + \text{locally cyclic}
%\]
%where the first two components make up \textcolor{blue}{transitive subspace} and
%
%\begin{itemize}
%\item gradient flow is integrable to give a global ranking
%
%\item example (b) is locally (triangularly) acyclic, but cyclic on large scale
%
%\item example (a) is locally (triangularly) cyclic
%\end{itemize}
%}
%
%%\frame{
%
%%  \frametitle{Global, Local, and Pairwise Rankings}
%
%%  \begin{itemize}
%%  \item<1-> \textcolor{red}{Global} ranking is a function on $V$, $f: V\to \R$ 
%%  \item<1-> \textcolor{red}{Local} (partial) ranking: restriction of global ranking on a subset $U$, $f^\prime: U\to \R$
%%  \item<1-> \textcolor{red}{Pairwise} ranking: $g: V\times V\to \R$ (with $g_{ij}=-g_{ji}$)
%%  \subitem Note: pairwise rankings are simply \textcolor{red}{skew-symmetric} matrices $\ASL(n)$ or certain equivalence classes in $\ASL(n)$. Also we may view pairwise rankings as \textcolor{red}{weighted digraphs}.
%%  \end{itemize}
%%}
%
%%\frame{
%
%%  \frametitle{Why Pairwise Ranking?}
%
%%  \begin{itemize}
%%  \item Human mind can't make preference judgements on moderately large sets (e.g.\ no more than $7\pm 2$ in psychology study)
%%  \item But human can do pairwise comparison more easily and accurately
%%  \item Pairwise ranking naturally arises in tournaments, exchange Economics, etc.
%%  \item Pairwise ranking may reduce the bias caused by the arbitrariness of rating scale
%%  \item Pairwise ranking may contain \textcolor{red}{more information} than global ranking (to be seen soon)!
%%  \end{itemize}
%%}
%
%
%
%\subsection{Discrete Exterior Calculus and Combinatorial Laplacian}
%
%%\frame{
%%  \frametitle{Our Main Theme}
%%  Below we'll outline an approach to analyze 
%%  \begin{itemize}
%%  \item \textcolor{red}{cardinal}, and
%%  \item \textcolor{red}{pairwise} 
%%  \end{itemize}
%%  rankings, in a perspective from discrete calculus. 
%%  
%%  \medskip
%%  
%%  In a brief, we'll reach an \textcolor{red}{orthogonal decomposition} of pairwise rankings, by \textcolor{red}{Hodge Theory},
%%  \[ \textcolor{red}{\text{pairwise} = \text{global} + \text{consistent cyclic} + \text{inconsistent cyclic} }\]
%% }
%
%%\frame{\tableofcontents[current]}
%
%\frame{
%  \frametitle{Clique Complex of a Graph}
%Extend graph $G=(V,E)$ to a \textcolor{red}{simplicial complex} $\mathcal{K}(G)$ by
%attaching triangles
%  \begin{itemize}
%  \item 0-simplices $\mathcal{K}_0(G)$: $V$
%  \item 1-simplices $\mathcal{K}_1(G)$: $E$
%  \item 2-simplices $\mathcal{K}_2(G)$: 3-cliques $\{i,j,k\}$ such that every edge exists in $E$
%  \item $k$-simplices $\mathcal{K}_{k}(G)$: $(k+1)$-cliques $\{i_{0},\dots
%,i_{k}\}$ of $G$
%  \end{itemize}
%  \textcolor{blue}{Note}: it suffices here to construct $\mathcal{K}(G)$ up to dimension \textcolor{red}{2}!
%}
%
%\frame{
%
%  \frametitle{Discrete $k$-forms or Cochains on $\mathcal{K}(G)$}
%   
%  \begin{itemize} 
%  \item  \textcolor{red}{$k$-forms or cochains} \[
%C^{k}(\mathcal{K}(G),\mathbb{R})=\{u:\mathcal{K}_{k+1}(G)\rightarrow
%\mathbb{R},u_{i_{\sigma(0)},\dots,i_{\sigma(k)}}=\operatorname*{sign}%
%(\sigma)u_{i_{0},\ldots,i_{k}}\}
%\]
%for $(i_{0},\ldots,i_{k})\in\mathcal{K}_{k+1}(G)$, where $\sigma
%\in\mathfrak{S}_{k+1}$ is a permutation on $(0,\dots,k)$.
%  \subitem global ranking: 0-forms $v\in C^0(K,\R)\cong \R^{n}$
%  \subitem pairwise ranking: 1-forms $w\in C^1(K,\R)$, $w_{ij}=-w_{ji}$
%  \item May put metrics/inner products on $C^{k}(\mathcal{K}(G),\mathbb{R})$, e.g. the following metric on $1$-forms for imbalance issue
%\[
%\left\langle w_{ij},\omega_{ij}\right\rangle _{D}=\sum_{(i,j)\in E}%
%D_{ij}w_{ij}\omega_{ij}%
%\]
%where
%\[
%D_{ij}=\lvert\{\text{customers who rate both $i$ and $j$}\}\rvert.
%\]
%  \end{itemize}   
%}
%
%\frame{
%
%  \frametitle{Coboundary Maps}
%  \begin{itemize}
%  \item $k$-dimensional \textcolor{red}{coboundary maps} $\delta_k : C^k(V,\R) \to C^{k+1}(V,\R)$ are defined as the \textcolor{red}{alternating difference} operator 
%  \[ (\delta_k u)(i_0,\ldots,i_{k+1}) = \sum_{j=0}^{k+1} (-1)^{j+1} u(i_0,\ldots,i_{j-1},i_{j+1},\ldots,i_{k+1}) \]
%  \item $\delta_k$ plays the role of \textcolor{red}{differentiation} 
%  \item $\delta_{k+1}\circ\delta_k = 0$
%  \item In particular,
%  \subitem $(\delta_{0} v) (i,j)= v_{j} - v_{i}=:(\operatorname*{\textcolor{red}{grad}} v)(i,j)$
%  \subitem  $(\delta_{1} w) (i,j,k) = (\pm)( w_{ij}+ w_{jk} + w_{ki})=:(\operatorname*{\textcolor{red}{curl}} w)(i,j,k)$ 
%  \end{itemize}
%}
%
%\frame{
%\frametitle{Combinatorial Curl and Divergence}
%For each triangle $\{i,j,k\}$, the \textcolor{red}{combinatorial curl}
%\[
%(\operatorname*{curl} w)(i,j,k) = (\delta_{1}w )(i,j,k) = w_{ij} + w_{jk} +
%w_{ki}
%\]
%measures the total flow-sum along the loop $i\to j\to k\to i$.
%
%\begin{itemize}
%\item $(\delta_{1}w )(i,j,k) = 0$ implies the flow is
%\textcolor{blue}{path-independent}, which defines the \textcolor{blue}{triangular transitivity
%subspace}.
%\end{itemize}
%
%For each alternative $i\in V$, the \textcolor{red}{combinatorial divergence}
%\[
%(\operatorname*{div} w)(i) := - (\delta_{0}^{T} w )(i) := \sum w_{i\ast}
%\]
%measures the \textcolor{blue}{inflow-outflow sum} at $i$.
%
%\begin{itemize}
%\item $(\delta_{0}^{T} w )(i)=0$ implies alternative $i$ is preference-neutral
%in all pairwise comparisons.
%
%\item divergence-free flow $\delta_{0}^{T} w =0$ is \textcolor{blue}{cyclic}
%\end{itemize}
%}
%
%%\frame{
%%  \frametitle{A View from Discrete Calculus}
%
%%We have the following cochain complex
%%\[
%%C^0(K,\R) \xrightarrow{\delta_0} C^1(K,\R) \xrightarrow{\delta_1}
%%C^2(K,\R),
%%\]
%
%%in other words,
%
%%\[ 
%%\textcolor{red}{Global \xrightarrow{\grad} Pairwise \xrightarrow{\curl} \text{Triplewise}}
%%\]
%%and 
%%\[ \textcolor{red}{\curl \circ \grad (Global\ rankings) = 0} \]
%%} 
%
%%\frame{
%%  \frametitle{What does it tell us?}
%%\[ 
%%\textcolor{red}{Global \xrightarrow{\grad} Pairwise \xrightarrow{\curl} \text{3-alternating tensors}}
%%\]
%%  \begin{itemize}
%%  \item \textcolor{red}{$\grad(Global) $} (i.e. $\im(\delta_0)$): a \textcolor{blue}{proper} subset of pairwise rankings induced from global
%%  \item \textcolor{red}{$\curl(Pairwise)$} (i.e. $\im(\delta_1)$): measures the \textcolor{blue}{consistency}/triangular arbitrage on triangle $\{i,j,k\}$
%%  \[  (\delta_1 g) (i,j,k)=g_{ij}+g_{jk}+g_{ki}  \]
%%  \subitem \textcolor{red}{$\ker(\curl)$} (i.e. $\ker(\delta_1)$): \textcolor{blue}{consistent, curl-free, triangular arbitrage-free}, in particular ---
%%  \subitem \textcolor{red}{$\curl\circ \grad (Global)=0$} (i.e. $\delta_1\circ\delta_0=0$) says global rankings are consistent/curl-free
%%\end{itemize}
%%}
%
%%\frame{
%%  \frametitle{Reverse direction: conjugate operators}
%%  \[ \textcolor{red}{Gradient \xleftarrow{\grad^\ast (=:-\dive)} Pairwise \xleftarrow{curl^\ast} \text{3-alternating}\ C^2(K,\R) } \]
%%  \begin{itemize}
%%  \item \textcolor{red}{$\grad^\ast$}: $\delta_0^T$ under Euclidean inner product, gives the total \textcolor{blue}{inflow-outflow difference} at each vertex (\textcolor{red}{negative divergence})
%%  \[ (\delta_0^T g)(i) = \sum g_{\ast i}- \sum g_{i\ast} \]
%%  \subitem $\ker(\delta_0^T)$, as \textcolor{red}{divergence-free}, is \textcolor{red}{cyclic} (interior/boundary)
%%  \item \textcolor{red}{$\curl^\ast$}: $\delta_1^T$, gives \textcolor{blue}{interior cyclic} pairwise rankings along triangles in $K_2$, which are \textcolor{blue}{inconsistent}
%%  \end{itemize}
%%}
%
%\subsection{Combinatorial Laplacian Operator}
%\frame{
%  \frametitle{Combinatorial Laplacian}
%  
%  \begin{itemize}
%  \item Define the $k$-dimensional \textcolor{red}{combinatorial Laplacian}, $\Delta_k:C^k \to C^k$ by
%\[ \Delta_k = \delta_{k-1} \delta_{k-1}^T + \delta_k^T \delta_k, \qquad k>0 \]   
%  \item $k=0$, $\Delta_0 = \delta_0^T \delta_0$ is the well-known \textcolor{red}{graph Laplacian}
%  \item $k=1$, 
%\[ \textcolor{red}{\Delta_1 = \curl \circ \curl^\ast - \dive \circ \grad} \]
%  \item Important Properties: 
%  \subitem $\Delta_k$ positive semi-definite
%  \subitem $\ker(\Delta_k)=\ker(\delta_{k-1}^T)\cap\ker(\delta_k)$: $k$-\textcolor{red}{Harmonics}, dimension equals to $k$-th Betti number
%  \subitem Hodge Decomposition Theorem
%  \end{itemize}
%}
%
%\subsection{Hodge Theory}
%\frame{
%  \frametitle{Hodge Decomposition Theorem}
%
%\begin{theorem}
%The space of $k$-forms (cochains) $C^k(\mathcal{K}(G),\R)$, admits an orthogonal decomposition into three
%\[
%C^k(\mathcal{K}(G),\R) = \im (\delta_{k-1}) \oplus H_k \oplus \im(\delta_k^T)
%\]
%where
%\[
%H_k = \ker(\delta_{k-1}) \cap \ker(\delta_k^T) = \ker (\Delta_k).
%\]
%\end{theorem}
%}
%
%\frame{
%  \frametitle{In particular $k=1$ for pairwise rankings}
%  \begin{figure}
%  \includegraphics[width=0.7\textwidth]{figures/Hodge2.pdf}
%  \caption{Hodge Decomposition for Pairwise Rankings}
%  \end{figure}
%}
%
%\frame{
%\frametitle{Harmonic rankings: locally consistent but globally inconsistent}
%\begin{columns}
%\begin{column}{0.5\textwidth}
%\begin{figure}
%%\resizebox{3cm}{!}{\input{figures/harmonic.pstex_t}}
%\includegraphics[width=0.8\textwidth]{figures/xiaoye05.pdf}
%\caption{A locally consistent but globally cyclic harmonic ranking.}
%\end{figure}
%\end{column}
%\begin{column}{0.5\textwidth}
%\begin{figure}
%\includegraphics[width=0.8\textwidth]{figures/harm25_tmp.pdf}
%%\includegraphics[width=0.8\textwidth]{figures/harm25_tmp.pdf}
%\caption{A harmonic ranking from truncated Netflix movie-movie network}
%\end{figure}
%\end{column}
%\end{columns}%
%}
%
%%\frame{
%%  \frametitle{An Example from Jester Dataset}
%%  \begin{figure}
%%  \includegraphics[width=0.9\textwidth]{figures/Hodge-Jester.pdf}
%%  \caption{Hodge Decomposition for a pairwise ranking on four Jester jokes (No.1 - 4): $\hat{g}_1$ gives
%%  a global ranking (order: $1>2>3>4$) which accounts for $90\%$ of the total norm; $\hat{g}_2$ is the consistent cyclic part on triangles $\{\{123\},\{124\}\}$ with $7\%$ norm; and $\hat{g}_3$ is the inconsistent cyclic part. }
%%  \end{figure}
%%}
%
%\frame{
%  \frametitle{Rank Aggregation as a Linear Projection}
%\begin{corollary}
%Every pairwise ranking admits a unique orthogonal decomposition,
%\[
%w = \proj_{\im \de_0} w + \proj_{\ker(\de_0^T)} w
%\]
%i.e. 
%\[ \textcolor{red}{pairwise} = \textcolor{red}{grad(global)} + \textcolor{red}{cyclic} \]
%\end{corollary}
%Particularly the first projection $\operatorname*{grad}($global$)$ gives a
%global ranking
%\[
%x^{\ast}=(\delta_{0}^{\ast}\delta_{0})^{\dag}\delta_{0}^{\ast}w=-(\Delta
%_{0})^{\dag}\operatorname*{div}(w)
%\]
%which extends the \textcolor{red}{Borda Count} as the best rule in classical social choice and voting theory (Saari).
%}
%
%\frame{
%  \frametitle{Saari's Geometric Explanation on Different Projections}
%\begin{figure}
%%\resizebox{3cm}{!}{\input{figures/harmonic.pstex_t}}
%\includegraphics[width=0.8\textwidth]{figures/Saari_projection.png}
%%\caption{A locally consistent but globally cyclic harmonic ranking.}
%\end{figure}
%}
%
%
%\frame{
%  \frametitle{Measuring Inconsistency by Curls}
%  \begin{definition}
%Define the \textcolor{red}{cyclicity ratio} of $w$ by  
%$$ C_p(w)  = \left(\frac{\| \proj_{\im (\curl^*)}  w \| }{\|w \|}\right)^2\leq 1$$
% \end{definition}
% 
%\textcolor{red}{Note}:
%\begin{itemize}
%\item This measures the total inconsistency within the data and pairwise ranking model $w$.
%\item Similarly we can define curls on each triangle.
%\end{itemize}
%}
%
%\frame{
%  \frametitle{Model Selection by Cyclicity Ratio (Curls)}
%\begin{figure}
%%\resizebox{3cm}{!}{\input{figures/harmonic.pstex_t}}
%\includegraphics[width=0.9\textwidth]{figures/6movies_comparison.pdf}
%\caption{MRQE: Movie-Review-Query-Engine (http://www.mrqe.com/)}
%\end{figure}
%}
%
%%\frame{
%%  \frametitle{Consistent Decomposition}
%%\begin{corollary}
%%(A) A consistent pairwise ranking $g$ associated with $K$, has a unique orthogonal decomposition
%%\[ g = \proj_{\im \de_0} g + \proj_{H_1} g = \textcolor{red}{\grad(global)+Harmonic} \]
%%i.e. 
%%where \textcolor{red}{Harmonic} is cyclic on the ``holes'' of the complex $K$.
%%(B)  Every consistent pairwise ranking on a contractable $K$, is induced from a global ranking.
%%\end{corollary}
%%\textcolor{blue}{Note}:
%%(B) rephrases the famous theorem in exchange Economics: \textcolor{red}{triangular arbitrage-free} implies \textcolor{red}{arbitrage-free} and the existence of \textcolor{red}{universal equivalent}.
%%}
%\frame{
%\frametitle{Erd\H{o}s-R\'{e}nyi random graph}%
%%EndExpansion
%
%Heuristical justification from Erd\H{o}s-R\'{e}nyi random graphs.
%
%\begin{theorem}
%[Kahle '07]For an Erd\H{o}s-R\'{e}nyi random graph $G(n,p)$ with $n$ vertices
%and edges forming independently with probability $p$, its clique complex
%$\chi_{G}$ will have zero $1$-homology almost always, except when
%\[
%\frac{1}{n^{2}}\ll p\ll\frac{1}{n}.
%\]
%\end{theorem}
%Since the full Netflix movie-movie comparison graph is almost complete
%($0.22\%$ missing edges), one may expect the chance of nontrivial harmonic
%ranking is small.
%}
%
%\subsection{Conclusions}
%\frame{
%  \frametitle{Conclusions}
%  Conclusions
%  \begin{itemize}
%  \item Rank Aggregation as \textcolor{red}{1-dimensional scaling} of data
%  \item Hodge Theory provides an \textcolor{red}{orthogonal decomposition} for pairwise rankings  
%  \item Such decomposition is helpful to characterize the \textcolor{red}{cyclicity} and \textcolor{red}{inconsistency} of pairwise rankings, as well as natural extension of Borda count as rank aggregation
%  algorithm
%  \end{itemize}
%}

\frame{
  \frametitle{Reference}
  \begin{itemize}
  \item Edelsbrunner, Letscher, and Zomorodian (2002) Topological Persistence and Simplification.   
  \item Ghrist, R. (2007) Barcdes: the Persistent Topology of Data. \emph{Bulletin of AMS}, 45(1):61-75. 
  \item Edelsbrunner, Harer (2008) Persistent Homology - a survey. \emph{Contemporary Mathematics}.
  \item Carlsson, G. (2009) Topology and Data. \emph{Bulletin of AMS}, 46(2):255-308.
  \item Camara et al. (2016) Topological Data Analysis Generates High-Resolution, Genome-wide Maps of Human Recombination, \emph{Cell Systems}, 3(1): 83–94.
  \item Wei, Guowei, (2017) Persistent Homology Analysis of Biomolecular Data, \emph{SIAM News}.

  \end{itemize}
}

\end{document}

